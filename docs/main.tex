\documentclass{article}
\usepackage[utf8]{inputenc}
\usepackage{listings}
\usepackage{xcolor}
\usepackage{geometry}
\usepackage{amsmath}
\usepackage{hyperref}
\usepackage{graphicx}
\usepackage{fancyhdr}
\usepackage{longtable}
\usepackage{booktabs}

% Configuración del diseño
\geometry{margin=1in}

% Eliminar sangría en todo el documento
\setlength{\parindent}{0pt}

% Encabezado y pie de página
\pagestyle{fancy}
\fancyhf{}
\fancyhead[L]{Bases de Datos II}
\fancyhead[R]{Proyecto 3}
\fancyfoot[C]{\thepage}

% Configuración de colores para el código
\lstset{
    language=Python,
    basicstyle=\ttfamily\small,
    keywordstyle=\color{blue},
    commentstyle=\color{gray},
    stringstyle=\color{red},
    breaklines=true,
    frame=single,
    showstringspaces=false
}

\begin{document}

% Portada
\begin{titlepage}
    \centering
    {\Large\textbf{Tecnológico de Costa Rica}}\\[0.5cm]
    {\Large\textbf{Escuela de Ingeniería en Computación}}\\[0.5cm]
    {\Large\textbf{IC4302 Bases de Datos II}}\\[0.5cm]
    {\Large\textbf{Grupo 20}}\\[2cm]

    {\Large\bfseries Proyecto 3}\\[0.5cm]
    {\Large\bfseries \emph{SparkSQL}}\\[2cm]

    \textbf{\large Profesor:}\\
    \large Alberto Shum Chan \\[1.5cm]

    \textbf{\large Estudiantes:}\\[0.5cm]
    \large Andrés Felipe Arias Corrales\\
    2015028947\\[0.5cm]
    \large Diego Esteban Castro Chaves\\
    200419896\\[0.5cm]
    \large Fabián José Fernández Fernández\\
    2022144383\\[0.5cm]
    \large Hengerlyn D Angiely Araya Nash\\
    2021051418\\[2cm]

    \textbf{\large Fecha de Entrega:} 15 de Noviembre del 2024\\[0.5cm]
    \textbf{\large II semestre, 2024}
\end{titlepage}

% Índice
\newpage
\tableofcontents
\newpage

% Sección 1: Descripción General del Sistema
\section{Descripción General del Sistema}

El objetivo principal de este sistema es realizar un análisis y visualización de datos provenientes de diferentes fuentes, específicamente del Instituto Nacional de Estadística y Censos (INEC) y del Organismo de Investigación Judicial (OIJ). A través de procesos de extracción, transformación y carga (ETL), se busca integrar estos conjuntos de datos para analizar la relación entre factores socioeconómicos y criminalidad en Costa Rica.

\subsection{Componentes del Sistema}

\begin{itemize}
    \item \textbf{Extracción de Datos:} Obtención de los datos de los archivos CSV del OIJ y del INEC.
    \item \textbf{Limpieza y Preprocesamiento:} Incluye normalización de nombres de distritos, eliminación de inconsistencias, conversión de formatos, y eliminación de duplicados.
    \item \textbf{Análisis de Datos:} Comparación de distritos y correlación entre estadísticas criminales y socioeconómicas.
    \item \textbf{Visualización:} Gráficos y mapas de calor para explorar patrones.
    \item \textbf{Almacenamiento:} Uso de PostgreSQL para almacenar los datos procesados.
\end{itemize}

\subsection{Tecnologías Utilizadas}

\begin{itemize}
    \item \textbf{Python:} Para la manipulación de datos y la creación de gráficos.
    \item \textbf{PySpark:} Para el procesamiento de grandes volúmenes de datos.
    \item \textbf{PostgreSQL:} Para el almacenamiento de datos procesados.
    \item \textbf{Matplotlib y Seaborn:} Para la generación de gráficos detallados.
\end{itemize}

\subsection{Objetivos del Proyecto}

\begin{itemize}
    \item Identificar patrones y tendencias en la actividad criminal.
    \item Relacionar indicadores socioeconómicos con los índices de criminalidad.
    \item Proveer un sistema escalable y eficiente para manejar grandes volúmenes de datos.
\end{itemize}

% Sección 2: Funciones del Proyecto
\section{Funciones del Proyecto}

\subsection{Conexión a PostgreSQL}

Esta función establece la conexión con la base de datos PostgreSQL para almacenar y consultar los datos procesados.

\begin{lstlisting}[language=Python, caption={Función para establecer conexión con PostgreSQL}]
import psycopg

def conectar_postgresql(conn_params):
    """
    Conecta a PostgreSQL utilizando los parámetros de conexión.
    Args:
        conn_params (dict): Diccionario con los parámetros de conexión.
    Returns:
        Connection: Objeto de conexión a PostgreSQL.
    """
    try:
        conn = psycopg.connect(**conn_params)
        print("Conexión exitosa a PostgreSQL")
        return conn
    except Exception as e:
        print(f"Error al conectar a PostgreSQL: {e}")
        return None
\end{lstlisting}

\subsection{Limpieza de Espacios y Normalización}

\begin{lstlisting}[language=Python, caption={Función para eliminar espacios y normalizar datos}]
from pyspark.sql.functions import regexp_replace, lower

def limpiar_distritos(dataframe, columna):
    """
    Limpia espacios y normaliza los nombres de los distritos.
    Args:
        dataframe: DataFrame de Spark.
        columna: Nombre de la columna a limpiar.
    Returns:
        DataFrame limpio y normalizado.
    """
    return dataframe.withColumn(columna, regexp_replace(lower(dataframe[columna]), " ", ""))
\end{lstlisting}

\subsection{Unión de Conjuntos de Datos}

\begin{lstlisting}[language=Python, caption={Función para unir datos del OIJ e INEC}]
def unir_datos(oij_df, inec_df, columna_oij, columna_inec):
    """
    Une los datos del OIJ e INEC basándose en distritos comunes.
    Args:
        oij_df: DataFrame del OIJ.
        inec_df: DataFrame del INEC.
        columna_oij: Nombre de la columna de distritos del OIJ.
        columna_inec: Nombre de la columna de distritos del INEC.
    Returns:
        DataFrame unido.
    """
    return oij_df.join(inec_df, oij_df[columna_oij] == inec_df[columna_inec])
\end{lstlisting}

% Visualizaciones
\section{Visualización de los Datos}

\subsection{Cantidad de Delitos por Distrito}

\begin{lstlisting}[language=Python, caption={Visualización de delitos por distrito}]
import matplotlib.pyplot as plt

# Agrupar datos
delitos_por_distrito = df_oij.groupBy("distrito").count().orderBy("count", ascending=False).limit(10)

# Convertir a Pandas para graficar
delitos_pd = delitos_por_distrito.toPandas()

# Crear gráfico
plt.figure(figsize=(10, 6))
plt.bar(delitos_pd['distrito'], delitos_pd['count'], color='skyblue')
plt.xlabel("Distrito")
plt.ylabel("Cantidad de Delitos")
plt.title("Delitos por Distrito")
plt.xticks(rotation=45)
plt.tight_layout()
plt.show()
\end{lstlisting}

\subsection{Mapa de Calor de Delitos por Hora y Tipo}

\begin{lstlisting}[language=Python, caption={Mapa de calor de delitos por hora y tipo}]
import seaborn as sns

# Preparar datos
heatmap_data = df_oij.groupBy("hora", "delito").count().toPandas().pivot(index="hora", columns="delito", values="count").fillna(0)

# Graficar
plt.figure(figsize=(12, 8))
sns.heatmap(heatmap_data, cmap="YlGnBu", annot=True, fmt="d")
plt.title("Mapa de Calor: Delitos por Hora y Tipo")
plt.xlabel("Tipo de Delito")
plt.ylabel("Hora")
plt.show()
\end{lstlisting}

\subsection{Comparación de Factores Socioeconómicos}

\begin{lstlisting}[language=Python, caption={Gráfico comparativo de factores socioeconómicos}]
# Comparar delitos con tasa de ocupación
merged_data = pd.merge(delitos_pd, tasa_ocupacion, on="distrito")

# Crear gráfico
plt.figure(figsize=(12, 6))
plt.bar(merged_data["distrito"], merged_data["delitos"], label="Delitos", alpha=0.7)
plt.plot(merged_data["distrito"], merged_data["tasa_ocupacion"], label="Tasa de Ocupación (%)", color="orange", marker="o")
plt.title("Comparación: Delitos y Tasa de Ocupación")
plt.xlabel("Distrito")
plt.ylabel("Cantidad/Tasa (%)")
plt.legend()
plt.xticks(rotation=45)
plt.tight_layout()
plt.show()
\end{lstlisting}

% Conclusiones
\section{Conclusiones}

\begin{enumerate}
    \item La limpieza y normalización de datos es esencial para una integración efectiva.
    \item Las visualizaciones generadas permiten identificar patrones relevantes en los datos.
    \item SparkSQL y PostgreSQL son herramientas complementarias que mejoran el análisis y almacenamiento de grandes volúmenes de datos.
    \item La correlación entre factores socioeconómicos y delitos refleja desigualdades estructurales.
\end{enumerate}

% Referencias
\section{Referencias}

\begin{itemize}
    \item Instituto Nacional de Estadística y Censos (INEC). \emph{Censo 2011: Indicadores socioeconómicos}.
    \item Organismo de Investigación Judicial (OIJ). \emph{Estadísticas de criminalidad 2011}.
\end{itemize}

\end{document}
